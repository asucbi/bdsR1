% Options for packages loaded elsewhere
\PassOptionsToPackage{unicode}{hyperref}
\PassOptionsToPackage{hyphens}{url}
%
\documentclass[
]{article}
\usepackage{amsmath,amssymb}
\usepackage{iftex}
\ifPDFTeX
  \usepackage[T1]{fontenc}
  \usepackage[utf8]{inputenc}
  \usepackage{textcomp} % provide euro and other symbols
\else % if luatex or xetex
  \usepackage{unicode-math} % this also loads fontspec
  \defaultfontfeatures{Scale=MatchLowercase}
  \defaultfontfeatures[\rmfamily]{Ligatures=TeX,Scale=1}
\fi
\usepackage{lmodern}
\ifPDFTeX\else
  % xetex/luatex font selection
\fi
% Use upquote if available, for straight quotes in verbatim environments
\IfFileExists{upquote.sty}{\usepackage{upquote}}{}
\IfFileExists{microtype.sty}{% use microtype if available
  \usepackage[]{microtype}
  \UseMicrotypeSet[protrusion]{basicmath} % disable protrusion for tt fonts
}{}
\makeatletter
\@ifundefined{KOMAClassName}{% if non-KOMA class
  \IfFileExists{parskip.sty}{%
    \usepackage{parskip}
  }{% else
    \setlength{\parindent}{0pt}
    \setlength{\parskip}{6pt plus 2pt minus 1pt}}
}{% if KOMA class
  \KOMAoptions{parskip=half}}
\makeatother
\usepackage{xcolor}
\usepackage[margin=1in]{geometry}
\usepackage{color}
\usepackage{fancyvrb}
\newcommand{\VerbBar}{|}
\newcommand{\VERB}{\Verb[commandchars=\\\{\}]}
\DefineVerbatimEnvironment{Highlighting}{Verbatim}{commandchars=\\\{\}}
% Add ',fontsize=\small' for more characters per line
\usepackage{framed}
\definecolor{shadecolor}{RGB}{248,248,248}
\newenvironment{Shaded}{\begin{snugshade}}{\end{snugshade}}
\newcommand{\AlertTok}[1]{\textcolor[rgb]{0.94,0.16,0.16}{#1}}
\newcommand{\AnnotationTok}[1]{\textcolor[rgb]{0.56,0.35,0.01}{\textbf{\textit{#1}}}}
\newcommand{\AttributeTok}[1]{\textcolor[rgb]{0.13,0.29,0.53}{#1}}
\newcommand{\BaseNTok}[1]{\textcolor[rgb]{0.00,0.00,0.81}{#1}}
\newcommand{\BuiltInTok}[1]{#1}
\newcommand{\CharTok}[1]{\textcolor[rgb]{0.31,0.60,0.02}{#1}}
\newcommand{\CommentTok}[1]{\textcolor[rgb]{0.56,0.35,0.01}{\textit{#1}}}
\newcommand{\CommentVarTok}[1]{\textcolor[rgb]{0.56,0.35,0.01}{\textbf{\textit{#1}}}}
\newcommand{\ConstantTok}[1]{\textcolor[rgb]{0.56,0.35,0.01}{#1}}
\newcommand{\ControlFlowTok}[1]{\textcolor[rgb]{0.13,0.29,0.53}{\textbf{#1}}}
\newcommand{\DataTypeTok}[1]{\textcolor[rgb]{0.13,0.29,0.53}{#1}}
\newcommand{\DecValTok}[1]{\textcolor[rgb]{0.00,0.00,0.81}{#1}}
\newcommand{\DocumentationTok}[1]{\textcolor[rgb]{0.56,0.35,0.01}{\textbf{\textit{#1}}}}
\newcommand{\ErrorTok}[1]{\textcolor[rgb]{0.64,0.00,0.00}{\textbf{#1}}}
\newcommand{\ExtensionTok}[1]{#1}
\newcommand{\FloatTok}[1]{\textcolor[rgb]{0.00,0.00,0.81}{#1}}
\newcommand{\FunctionTok}[1]{\textcolor[rgb]{0.13,0.29,0.53}{\textbf{#1}}}
\newcommand{\ImportTok}[1]{#1}
\newcommand{\InformationTok}[1]{\textcolor[rgb]{0.56,0.35,0.01}{\textbf{\textit{#1}}}}
\newcommand{\KeywordTok}[1]{\textcolor[rgb]{0.13,0.29,0.53}{\textbf{#1}}}
\newcommand{\NormalTok}[1]{#1}
\newcommand{\OperatorTok}[1]{\textcolor[rgb]{0.81,0.36,0.00}{\textbf{#1}}}
\newcommand{\OtherTok}[1]{\textcolor[rgb]{0.56,0.35,0.01}{#1}}
\newcommand{\PreprocessorTok}[1]{\textcolor[rgb]{0.56,0.35,0.01}{\textit{#1}}}
\newcommand{\RegionMarkerTok}[1]{#1}
\newcommand{\SpecialCharTok}[1]{\textcolor[rgb]{0.81,0.36,0.00}{\textbf{#1}}}
\newcommand{\SpecialStringTok}[1]{\textcolor[rgb]{0.31,0.60,0.02}{#1}}
\newcommand{\StringTok}[1]{\textcolor[rgb]{0.31,0.60,0.02}{#1}}
\newcommand{\VariableTok}[1]{\textcolor[rgb]{0.00,0.00,0.00}{#1}}
\newcommand{\VerbatimStringTok}[1]{\textcolor[rgb]{0.31,0.60,0.02}{#1}}
\newcommand{\WarningTok}[1]{\textcolor[rgb]{0.56,0.35,0.01}{\textbf{\textit{#1}}}}
\usepackage{graphicx}
\makeatletter
\newsavebox\pandoc@box
\newcommand*\pandocbounded[1]{% scales image to fit in text height/width
  \sbox\pandoc@box{#1}%
  \Gscale@div\@tempa{\textheight}{\dimexpr\ht\pandoc@box+\dp\pandoc@box\relax}%
  \Gscale@div\@tempb{\linewidth}{\wd\pandoc@box}%
  \ifdim\@tempb\p@<\@tempa\p@\let\@tempa\@tempb\fi% select the smaller of both
  \ifdim\@tempa\p@<\p@\scalebox{\@tempa}{\usebox\pandoc@box}%
  \else\usebox{\pandoc@box}%
  \fi%
}
% Set default figure placement to htbp
\def\fps@figure{htbp}
\makeatother
\setlength{\emergencystretch}{3em} % prevent overfull lines
\providecommand{\tightlist}{%
  \setlength{\itemsep}{0pt}\setlength{\parskip}{0pt}}
\setcounter{secnumdepth}{-\maxdimen} % remove section numbering
\usepackage{bookmark}
\IfFileExists{xurl.sty}{\usepackage{xurl}}{} % add URL line breaks if available
\urlstyle{same}
\hypersetup{
  pdftitle={Nobel winners - key file},
  pdfauthor={Nicholas Duran; Jessica Kosie},
  hidelinks,
  pdfcreator={LaTeX via pandoc}}

\title{Nobel winners - key file}
\author{Nicholas Duran; Jessica Kosie}
\date{}

\begin{document}
\maketitle

Code-Along File \#1

Here we will practice reading and writing csv files.

First, let take a look at our file structure. What files and folders do
we already have in our environment?

\begin{Shaded}
\begin{Highlighting}[]
\FunctionTok{library}\NormalTok{(tidyverse)}
\end{Highlighting}
\end{Shaded}

Read in the \texttt{nobel.csv} file from the \texttt{data-raw/} folder.

Take a look at the data file. What kind of file is it? What function
would you use to read in the file? What is the path to the file?

\begin{Shaded}
\begin{Highlighting}[]
\NormalTok{nobel }\OtherTok{\textless{}{-}} \FunctionTok{read\_csv}\NormalTok{(}\StringTok{"data{-}raw/nobel.csv"}\NormalTok{)}
\end{Highlighting}
\end{Shaded}

\begin{verbatim}
## Rows: 935 Columns: 26
## -- Column specification --------------------------------------------------------
## Delimiter: ","
## chr  (21): firstname, surname, category, affiliation, city, country, gender,...
## dbl   (3): id, year, share
## date  (2): born_date, died_date
## 
## i Use `spec()` to retrieve the full column specification for this data.
## i Specify the column types or set `show_col_types = FALSE` to quiet this message.
\end{verbatim}

Use the \texttt{glimpse()} function to take a look at the data. We have
used this dataset before, so it should look familiar.

\begin{Shaded}
\begin{Highlighting}[]
\FunctionTok{glimpse}\NormalTok{(nobel)}
\end{Highlighting}
\end{Shaded}

\begin{verbatim}
## Rows: 935
## Columns: 26
## $ id                    <dbl> 1, 2, 3, 4, 5, 6, 6, 8, 9, 10, 11, 12, 13, 14, 1~
## $ firstname             <chr> "Wilhelm Conrad", "Hendrik A.", "Pieter", "Henri~
## $ surname               <chr> "Röntgen", "Lorentz", "Zeeman", "Becquerel", "Cu~
## $ year                  <dbl> 1901, 1902, 1902, 1903, 1903, 1903, 1911, 1904, ~
## $ category              <chr> "Physics", "Physics", "Physics", "Physics", "Phy~
## $ affiliation           <chr> "Munich University", "Leiden University", "Amste~
## $ city                  <chr> "Munich", "Leiden", "Amsterdam", "Paris", "Paris~
## $ country               <chr> "Germany", "Netherlands", "Netherlands", "France~
## $ born_date             <date> 1845-03-27, 1853-07-18, 1865-05-25, 1852-12-15,~
## $ died_date             <date> 1923-02-10, 1928-02-04, 1943-10-09, 1908-08-25,~
## $ gender                <chr> "male", "male", "male", "male", "male", "female"~
## $ born_city             <chr> "Remscheid", "Arnhem", "Zonnemaire", "Paris", "P~
## $ born_country          <chr> "Germany", "Netherlands", "Netherlands", "France~
## $ born_country_code     <chr> "DE", "NL", "NL", "FR", "FR", "PL", "PL", "GB", ~
## $ died_city             <chr> "Munich", NA, "Amsterdam", NA, "Paris", "Sallanc~
## $ died_country          <chr> "Germany", "Netherlands", "Netherlands", "France~
## $ died_country_code     <chr> "DE", "NL", "NL", "FR", "FR", "FR", "FR", "GB", ~
## $ overall_motivation    <chr> NA, NA, NA, NA, NA, NA, NA, NA, NA, NA, NA, NA, ~
## $ share                 <dbl> 1, 2, 2, 2, 4, 4, 1, 1, 1, 1, 1, 1, 2, 2, 1, 1, ~
## $ motivation            <chr> "\"in recognition of the extraordinary services ~
## $ born_country_original <chr> "Prussia (now Germany)", "the Netherlands", "the~
## $ born_city_original    <chr> "Lennep (now Remscheid)", "Arnhem", "Zonnemaire"~
## $ died_country_original <chr> "Germany", "the Netherlands", "the Netherlands",~
## $ died_city_original    <chr> "Munich", NA, "Amsterdam", NA, "Paris", "Sallanc~
## $ city_original         <chr> "Munich", "Leiden", "Amsterdam", "Paris", "Paris~
## $ country_original      <chr> "Germany", "the Netherlands", "the Netherlands",~
\end{verbatim}

Split the data into two (STEM and non-STEM):

\begin{itemize}
\tightlist
\item
  Create a new data frame, \texttt{nobel\_stem}, that filters for the
  STEM fields: (Physics, Medicine, Chemistry, and Economics). Create
  another data frame, \texttt{nobel\_nonstem}, that filters for the
  remaining fields.
\end{itemize}

\textbf{Hint:} Use the \texttt{\%in\%} operator when
\texttt{filter()}ing.

\begin{Shaded}
\begin{Highlighting}[]
\CommentTok{\# stem laureates}
\NormalTok{stem }\OtherTok{\textless{}{-}}\NormalTok{ nobel }\SpecialCharTok{\%\textgreater{}\%}
  \FunctionTok{filter}\NormalTok{(category }\SpecialCharTok{\%in\%} \FunctionTok{c}\NormalTok{(}\StringTok{"Physics"}\NormalTok{, }\StringTok{"Chemistry"}\NormalTok{, }\StringTok{"Medicine"}\NormalTok{, }\StringTok{"Economics"}\NormalTok{))}

\CommentTok{\# non{-}steam laureates}
\NormalTok{non\_stem }\OtherTok{\textless{}{-}}\NormalTok{ nobel }\SpecialCharTok{\%\textgreater{}\%}
  \FunctionTok{filter}\NormalTok{(}\SpecialCharTok{!}\NormalTok{(category }\SpecialCharTok{\%in\%} \FunctionTok{c}\NormalTok{(}\StringTok{"Physics"}\NormalTok{, }\StringTok{"Chemistry"}\NormalTok{, }\StringTok{"Medicine"}\NormalTok{, }\StringTok{"Economics"}\NormalTok{)))}
\end{Highlighting}
\end{Shaded}

Save the two data frames to \texttt{nobel-stem.csv} and
\texttt{nobel\_nonstem.csv}, respectively, in the \texttt{data/} folder.

Hint: we use \texttt{read\_csv()} to ``read'' in a data file. What might
we use to ``write'' a data file? How do we tell R to put the file into
the \texttt{data/} folder?

\begin{Shaded}
\begin{Highlighting}[]
\FunctionTok{write\_csv}\NormalTok{(stem, }\StringTok{"data/nobel\_stem.csv"}\NormalTok{)}
\FunctionTok{write\_csv}\NormalTok{(non\_stem, }\StringTok{"data/nobel\_nonstem.csv"}\NormalTok{)}
\end{Highlighting}
\end{Shaded}

Now, check out the \texttt{data/} folder to see if your files saved
correctly.

\end{document}
